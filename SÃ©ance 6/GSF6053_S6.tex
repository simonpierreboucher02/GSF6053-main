\documentclass{beamer}
\usepackage[utf8]{inputenc}
\usepackage{graphicx}
\usetheme{default}
\usecolortheme{default}

\title[S03 Extensions MLS]{Section 03 : Extensions au modèle linéaire simple\\ (Séance 6)}
\subtitle{GSF-6053: Économétrie Financière}
\author[SP. Boucher]{Simon-Pierre Boucher\inst{1}}
\institute[Université Laval]
{
  \inst{1}%
  Département de finance, assurance et immobilier\\
  Faculté des sciences de l'administration\\
  Université Laval}
\date[Hiver 2022]{15 février 2022}

\begin{document}

\begin{frame}
  \titlepage
\end{frame}

\begin{frame}{Références}
\textbf{Obligatoires:}
\begin{itemize}
\item \textbf{Notes de cours:} Section 3 (Professeure: Marie-Hélène Gagnon)
\item \textbf{Woolridge:} chapitres 3, 8, 12.
\end{itemize}
\vspace{0.5cm}
\textbf{Complémentaires:}
\begin{itemize}
\item \textbf{Gujarati et Porter:} chapitres 10, 11, 12, 13 et appendice C.
\item \textbf{Greene:} chapitres 2, 3, 4, 5, 9, 14, 20 C et D
\end{itemize}
\end{frame}


\begin{frame}{Plan de la séance}
  \tableofcontents
\end{frame}

\section{Hétéroscédasticité}

\frame{\tableofcontents[current]}


\begin{frame}{Hétéroscédasticité}
\begin{block}{Homoscédasticité}
\begin{itemize}
\item OLS makes the assumption that the variance of the error term is constant.
\begin{align*}
V(u_i)= \sigma^2
\end{align*}
\end{itemize}
\end{block}

\begin{block}{Hétéroscédasticité}
\begin{itemize}
\item Si les résidus n'ont pas une variance constante, on dit qu'ils sont hétéroscédastiques.
\begin{align*}
V(u_i)= \sigma_i^2
\end{align*}
\end{itemize}
\end{block}
\end{frame}


\begin{frame}{Hétéroscédasticité}
\begin{block}{Forme de $\Omega$}
\begin{itemize}
\item On connait rarement $\Omega$, il faut spécifier une forme pour cette matrice qui est gérable.
\item Estimation de tous les $\sigma_i^2$ pour chaque observations. 
\begin{align*}
\Omega= \begin{bmatrix} \sigma_1^2 & 0 & \cdots & 0 \\
0 & \sigma_{2}^2 & \cdots & 0 \\
\vdots & \vdots & \ddots & \vdots \\
0 & 0 & \cdots & \sigma_T^2
\end{bmatrix}
\end{align*}
\begin{itemize}
\item N’est pas une forme estimable, car le nombre de paramètres à estimer $T+K>T$.
\end{itemize}
\item Il faut donc spécifier un modèle (une paramétrisation) pour la forme de matrice $\Omega$ pour réduire le nombre de paramètres inconnus.
\end{itemize}
\end{block}
\end{frame}

\begin{frame}{Hétéroscédasticité}
\begin{block}{Hétéroscédasticité groupée}
\begin{itemize}
\item On a une variance hétéroscédastique, mais en sous-groupe.
\item À l’intérieur du sous-groupe, la variance est constante.
\end{itemize}
\begin{align*}
\sigma_t^2=\sigma_{10}^2, \hspace{0.5cm} t=1,...,T_1
\end{align*}

\begin{align*}
\sigma_t^2=\sigma_{20}^2, \hspace{0.5cm} t=T_{1+1},...,T
\end{align*}
\end{block}
\end{frame}


\begin{frame}{Hétéroscédasticité}
\begin{block}{Hétéroscédasticité groupée}
\begin{align*}
\Omega=\begin{bmatrix}
\begin{bmatrix}
\sigma_{10}^2 & 0 & \cdots  & 0 \\ 
0 & \sigma_{10}^2  & \cdots & 0 \\ 
\vdots & \vdots & \ddots & \vdots \\ 
0 & 0 & \cdots & \sigma_{10}^2
\end{bmatrix} & \begin{bmatrix}
0 & 0 & \cdots  & 0 \\ 
0 & 0  & \cdots & 0 \\ 
\vdots & \vdots & \ddots & \vdots \\ 
0 & 0 & \cdots & 0
\end{bmatrix} \\ \\
 \begin{bmatrix}
0 & 0 & \cdots  & 0 \\ 
0 & 0  & \cdots & 0 \\ 
\vdots & \vdots & \ddots & \vdots \\ 
0 & 0 & \cdots & 0
\end{bmatrix} & \begin{bmatrix}
\sigma_{20}^2 & 0 & \cdots  & 0 \\ 
0 & \sigma_{20}^2  & \cdots & 0 \\ 
\vdots & \vdots & \ddots & \vdots \\ 
0 & 0 & \cdots & \sigma_{20}^2
\end{bmatrix} 
\end{bmatrix}
\end{align*}
\end{block}
\end{frame}

\begin{frame}{Hétéroscédasticité}
\begin{block}{Hétéroscédasticité groupée}
On peut réécrire la matrice $\Omega$, avec hétéroscédasticité groupée comme suit:
\begin{align*}
\Omega=
\begin{bmatrix}
\sigma_{10}^2 I_{T_1} & 0\\ 
 0 & \sigma_{20}^2 I_{T_2}
\end{bmatrix}
\end{align*}
Sachant:
\begin{align*}
\sigma_{10}^2 I_{T_1}=\begin{bmatrix}
\sigma_{10}^2 & 0 & \cdots  & 0 \\ 
0 & \sigma_{10}^2  & \cdots & 0 \\ 
\vdots & \vdots & \ddots & \vdots \\ 
0 & 0 & \cdots & \sigma_{10}^2
\end{bmatrix} \textbf{et}\hspace{0.2cm} \sigma_{20}^2 I_{T_2}=\begin{bmatrix}
\sigma_{20}^2 & 0 & \cdots  & 0 \\ 
0 & \sigma_{20}^2  & \cdots & 0 \\ 
\vdots & \vdots & \ddots & \vdots \\ 
0 & 0 & \cdots & \sigma_{20}^2
\end{bmatrix}
\end{align*}
\end{block}
\end{frame}

\begin{frame}{Hétéroscédasticité}
\begin{block}{Hétéroscédasticité groupée}
Dans un cadre plus général:
\begin{align*}
\Omega= \begin{bmatrix} \sigma_{10}^2I_{T_1} & 0 & \cdots & 0 \\
0 & \sigma_{20}^2I_{T_2} & \cdots & 0 \\
\vdots & \vdots & \ddots & \vdots \\
0 & 0 & \cdots & \sigma_{G0}^2I_{T_G}
\end{bmatrix}
\end{align*}
\begin{align*}
\Omega^{-1}= \begin{bmatrix} \frac{1}{\sigma_{10}^2I_{T_1}} & 0 & \cdots & 0 \\
0 & \frac{1}{\sigma_{20}^2I_{T_2}} & \cdots & 0 \\
\vdots & \vdots & \ddots & \vdots \\
0 & 0 & \cdots & \frac{1}{\sigma_{G0}^2I_{T_G}}
\end{bmatrix}
\end{align*}
\end{block}
\end{frame}

\begin{frame}{Hétéroscédasticité}
\begin{block}{Hétéroscédasticité groupée}
Dans un cadre plus général:

\begin{align*}
P^{-1}= \begin{bmatrix} \frac{1}{\sigma_{10}I_{T_1}} & 0 & \cdots & 0 \\
0 & \frac{1}{\sigma_{20}I_{T_2}} & \cdots & 0 \\
\vdots & \vdots & \ddots & \vdots \\
0 & 0 & \cdots & \frac{1}{\sigma_{G0}I_{T_G}}
\end{bmatrix}
\end{align*}
\begin{itemize}
\item Les moindres carrés sont donc pondérés par la matrice $P^{-1}$, ce qui revient à pondérer par l’écart type qui est hétéroscédastique par groupe $G$.
\end{itemize}
\end{block}
\end{frame}


\begin{frame}{Hétéroscédasticité}
\begin{block}{Hétéroscédasticité groupée}
Modèle transformé:

\begin{align*}
\begin{bmatrix} 
\frac{Y_1}{\sqrt{Z_1}}\\
\frac{Y_2}{\sqrt{Z_2}}\\
\vdots \\
\frac{Y_T}{\sqrt{Z_T}}
\end{bmatrix}
= 
\begin{bmatrix} 
\frac{1}{\sqrt{Z_1}}\\
\frac{1}{\sqrt{Z_2}}\\
\vdots \\
\frac{1}{\sqrt{Z_T}}
\end{bmatrix}
\beta_0
+
\begin{bmatrix} 
\frac{X_{11}}{\sqrt{Z_1}}\\
\frac{X_{12}}{\sqrt{Z_2}}\\
\vdots \\
\frac{X_{1T}}{\sqrt{Z_T}}
\end{bmatrix}
\beta_1
+ \cdots + 
\begin{bmatrix} 
\frac{X_{K-11}}{\sqrt{Z_1}}\\
\frac{X_{K-12}}{\sqrt{Z_2}}\\
\vdots \\
\frac{X_{K-1T}}{\sqrt{Z_T}}
\end{bmatrix}
\beta_K 
+
\begin{bmatrix} 
\frac{u_1}{\sqrt{Z_1}}\\
\frac{u_2}{\sqrt{Z_2}}\\
\vdots \\
\frac{u_T}{\sqrt{Z_T}}
\end{bmatrix}
\end{align*}
\begin{itemize}
\item Remarquez que le modèle devient sans constante. Ceci a des implications entre autres pour le $R^2$ du modèle.
\end{itemize}
\end{block}
\end{frame}

\begin{frame}{Hétéroscédasticité}
\begin{block}{Variance varie en fonction de plusieurs variables exogènes}
Soit le modèle suivant:
\begin{align*}
Y_t=\beta_0 + \beta_1 X_{1t}+\cdots+ \beta_{K-1} X_{k-1t}+u_t
\end{align*}
\begin{align*}
\sigma_t^2=\alpha_0+\alpha_1 Z_{1t}+\cdots+\alpha_H Z_{Ht}+\epsilon_t
\end{align*}
\end{block}
\end{frame}


\begin{frame}{Hétéroscédasticité}
\begin{block}{Variance varie en fonction de plusieurs variables exogènes}
\begin{itemize}
\item Cette paramétrisation de la variance demande d’estimer $\hat{\sigma}_t^2$ par une régression artificielle. 
\begin{itemize}
\item On utilise les résidus au carré, $\hat{u}^2$, pour avoir une proxy de $\sigma_t^1$.
\item On régresse les $\hat{u}^2$ sur les variables exogènes pour trouver les
paramètres $\alpha_h$ définissant la variance.
\item On trouve les valeurs des $\hat{\sigma}_t^2$ estimés par le modèle avec les résultats de
la régression auxiliaire :
\begin{align*}
\hat{\sigma}_t^2=\hat{\alpha}_0+\hat{\alpha}_1 Z_{1t}+\cdots+\hat{\alpha}_HZ_{Ht}
\end{align*}
\end{itemize}
\item Cette technique demande des données supplémentaires sur les exogènes.
\end{itemize}
\end{block}
\end{frame}


\begin{frame}{Hétéroscédasticité}
\begin{block}{Variance varie en fonction de plusieurs variables exogènes}
Le modèle transformé sera:
\begin{align*}
\frac{Y_t}{\hat{\sigma}_T}=\beta_0 \frac{1}{\hat{\sigma}_t}+\beta_1 \frac{X_{1t}}{\hat{\sigma}_t}+\cdots+\beta_{K-1} \frac{X_{K-1t}}{\hat{\sigma}_t}+\frac{u_t}{\hat{\sigma}_t}
\end{align*}
Où 
\begin{align*}
\hat{\sigma}=\sqrt{\hat{\alpha}_0+\hat{\alpha}_1 Z_{1t}+\cdots+\hat{\alpha}_H Z_{Ht}}
\end{align*}
En estimant par OLS
\begin{itemize}
\item Contrairement au cas à une seule variable exogène, les estimateurs des $\beta_K$ sont obtenus par FGLS et nom par GLS, car nous avons estimé la matrice $\Omega$ à partir de variable exogène.
\end{itemize}
\end{block}
\end{frame}

\section{Tests diagnostiques d'hétéroscédasticité}

\frame{\tableofcontents[current]}


\begin{frame}{Tests diagnostiques d'hétéroscédasticité}
\begin{itemize}
\item Nous introduisons des tests pour savoir si les régressions considérées souffrent de problème d’hétéroscédasticité des erreurs (et résidus). 
\item Une approche intuitive serait de regarder un graphique des résidus de régression dans le but de trouver des anomalies.
\item Une approche par analyse de graphique n’est pas un test statistique formel de la présence d’hétéroscédasticité.
\end{itemize}
\end{frame}

\begin{frame}{Tests diagnostiques d'hétéroscédasticité}
\begin{block}{Test de Goldfeld-Quandt (GQ)}
\textbf{Modèle:}
\begin{align*}
Y_i=\beta_0+\beta_1 X_{1i}+\cdots+\beta_{K-1}X_{K-1i}+u_i
\end{align*}
\begin{itemize}
\item On teste l’hypothèse nulle d’une variance constante dans le temps contre l’alternative que la variance augmente de manière monotone selon un exogène $Z_{1i}$.
\item Hypothèse nulle:
\begin{align*}
H_0: \sigma_i^2 = \sigma^2 \forall i
\end{align*}
\item Hypothèse alternative:
\begin{align*}
H_A: \sigma_i^2 \hspace{0.2cm} \textbf{augmente de manière monotone selon} \hspace{0.2cm} Z_{1i}
\end{align*}
\end{itemize}
\end{block}
\end{frame}

\begin{frame}{Tests diagnostiques d'hétéroscédasticité}
\begin{block}{Test de Goldfeld-Quandt (GQ)}
\textbf{Marche à suivre:}
\begin{enumerate}
\item Ordonner les observations selon la variable Z.
\item Omettre c observation centrale
\item Effectuer deux régressions séparées sur les $\frac{N-c}{2}$ premières et les $\frac{N-c}{2}$ dernières observations et calculer $\hat{\sigma}_1^2$ et $\hat{\sigma}_2^2$
\item La statistique de test est:
\begin{align*}
GQ = \frac{\hat{\sigma}_2^2}{\hat{\sigma}_1^2} \sim F \left( \frac{N-c-2K}{2}, \frac{N-c-2K}{2}\right)
\end{align*}
\end{enumerate}
\end{block}
\end{frame}


\begin{frame}{Tests diagnostiques d'hétéroscédasticité}
\begin{block}{Test de Goldfeld-Quandt (GQ)}
\begin{itemize}
\item les estimateurs de $\hat{\sigma}_1^2$ et de $\hat{\sigma}_2^2$ sont calculés à partir des deux régressions en sous-groupe. 
\item Le nombre d’observations sera $\frac{N-c}{•2}$ et le nombre de degrés de liberté sera $\frac{N-c}{2-k}$.
\item Ceci est équivalent à $$\frac{N-c-2K}{2}$ qui représente les degrés de liberté du numérateur et du dénominateur de la statistique.
\end{itemize}
\end{block}
\end{frame}

\begin{frame}{Tests diagnostiques d'hétéroscédasticité}
\begin{block}{Test de Goldfeld-Quandt (GQ)}
\begin{itemize}
\item Le test est exact et non asymptotique.
\item On perd les observations du centre, car on veut voir ce qui se passe aux extrémités. 
\item Si les erreurs sont normales, le test suit exactement une loi F et non asymptotiquement. 
\item On n’a donc pas besoin d’une grande taille d’échantillon pour ce test.
\end{itemize}
\end{block}
\end{frame}


\begin{frame}{Tests diagnostiques d'hétéroscédasticité}
\begin{block}{Test de Breusch-Pagan (BP)}
\begin{itemize}
\item Le test est plus général que le précédent et plus utilisé
\item En présence d'exogènes:
\begin{align*}
Y_t=\beta_0+\beta_1 X_{1t}+ \cdots+\beta_{K-1} X_{K-1t}+u_t
\end{align*}
\begin{align*}
\sigma_t^2=\alpha_0+\alpha_1 Z_{1t}+\cdots+\alpha_H Z_{Ht}+\epsilon_t
\end{align*}
\begin{align*}
H_0: \alpha_1=\cdots=\alpha_H=0 \forall h
\end{align*}
\begin{align*}
H_A: \exists \alpha_h \neq 0
\end{align*}
\end{itemize}
\end{block}
\end{frame}

\begin{frame}{Tests diagnostiques d'hétéroscédasticité}
\begin{block}{Test de Breusch-Pagan (BP)}
\begin{itemize}
\item Le test est basé sur une régression artificielle suivante :

\begin{align*}
u_t^2=\hat{\alpha_0}+\hat{\alpha_1} Z_{1t}+\cdots+\hat{\alpha_H} Z_{Ht}
\end{align*}
\item Le test porte sur tous les coefficients de la régression, sauf la constante. 
\item Il est donc naturel qu’il soit basé sur le R2 de la régression artificielle.
\end{itemize}
\end{block}
\end{frame}


\begin{frame}{Tests diagnostiques d'hétéroscédasticité}
\begin{block}{Test de Breusch-Pagan (BP)}
\textbf{Statistique de test:}
\begin{align*}
BP=T \times R_{reg artificielle}^2 
\end{align*}
\begin{align*}
BP^{asy} \sim X^{2}(H)
\end{align*}
\begin{itemize}
\item Le $R^2$ de la régression artificielle sera petit, mais cela ne veut pas dire qu’il n’y a pas d’hétéroscédasticité, car le test est basé sur $(T \times R^2)$.
\end{itemize}
\end{block}
\end{frame}

\begin{frame}{Tests diagnostiques d'hétéroscédasticité}
\begin{block}{Test d’effet ARCH sur les résidus de régression.}
\begin{itemize}
\item \textbf{ARCH:} autoregressive conditionnal heteroscedasticity
\item On veut tester l’hypothèse de \textbf{volatility clostering}, la variabilité d'aujourd’hui est reliée à la variabilité d'hier.

\begin{align*}
u_t = \epsilon_t \sqrt{\alpha_0+\alpha_1 u_{t-1}^2}
\end{align*}
Sachant
\begin{align*}
\epsilon_t \sim N(0,1)
\end{align*}
\end{itemize}
\end{block}
\end{frame}

\begin{frame}{Tests diagnostiques d'hétéroscédasticité}
\begin{block}{Test d’effet ARCH sur les résidus de régression.}
\begin{itemize}
\item \textbf{Variance conditionnelle au passé:}
\begin{align*}
E(u_t^2 \mid u_{t-1})=\alpha_0+\alpha_1 u_{t-1}^2
\end{align*}
\item La variance conditionnelle au passé est hétéroscédastique (varie dans le temps ici)
\item $V(u_t)$ (la variance inconditionnelle) est invariante dans le temps si $\mid \alpha_1 \mid < 1$
\end{itemize}
\end{block}
\end{frame}

\begin{frame}{Tests diagnostiques d'hétéroscédasticité}
\begin{block}{Test d’effet ARCH sur les résidus de régression.}
\begin{itemize}
\item \textbf{Variance inconditionnelle:}
\begin{align*}
E(u_t^2) = E(\epsilon_t^2) \times E(\alpha_0+\alpha_1 u_{t-1}^2)+cov(\epsilon_t,u_{t-1})
\end{align*}
\begin{align*}
V(u_t)=V(u)= \frac{\alpha_0}{1-\alpha_1}
\end{align*}
\item Pour évaluer cette hypothèse, on peut appliquer un test de Breusch-Pagan basé sur une régression artificielle des $\hat{u}_t^2$sur une constante et $\hat{u}_{t-1}^2$
\end{itemize}
\end{block}
\end{frame}

\begin{frame}{Tests diagnostiques d'hétéroscédasticité}
\begin{block}{Test de White}
\begin{itemize}
\item Ce test s’applique pour tester l’hétéroscédasticité de forme générale.
\item On ne spécifie pas de forme pour la variance.
\item L’hypothèse nulle est que la variance est homoscédastique contre une alternative hétéroscédastique générale.
\end{itemize}
Soit le modèle:
\begin{align*}
Y_t=\beta_0+\beta_1 X_{1t}+\beta_2 X_{2t}+u_t
\end{align*}
Le test est basé sur une régression artificielle:
\begin{align*}
\hat{u}_t^2=\alpha_0+\alpha_1 X_{1t}+\alpha_{11} X_{1t}^2+\alpha_2 X_{2t}+\alpha_{22} X_{2t}^2+\alpha_{12} X_{1t}X_{2t}+\epsilon_t
\end{align*}
\end{block}
\end{frame}

\begin{frame}{Tests diagnostiques d'hétéroscédasticité}
\begin{block}{Test de White}
\textbf{Statistique de test:}
\begin{align*}
WHITE = T \times R_{reg.art}^2
\end{align*}
\begin{align*}
WHITE^{asy} \sim X^2(H)
\end{align*}
\begin{itemize}
\item L’avantage du test de White est qu’il est très général.
\item Par contre, on perd des degrés de liberté de la régression artificielle, car on doit estimer les produits croisés et les régresseurs au carré.
\end{itemize}
\end{block}
\end{frame}

\begin{frame}{Tests diagnostiques d'hétéroscédasticité}
\begin{block}{Le Correction de White}
\begin{itemize}
\item Lorsque q’on ne peut spécifier une forme paramétrique pour la variance hétéroscédastique des erreurs, on ne peut pas utiliser les estimateurs GLS et FGLS définis plus haut. 
\item Une approche alternative est d’utiliser l’estimateur de $\hat{\beta}_{OLS}$ , mais de corriger l’estimateur de sa variance pour tenir compte de l’hétéroscédasticité. 
\item On se souvient que $\hat{\beta}_{OLS}$ était toujours sans biais pour les cas hétéroscédastiques, mais que sa variance était fausse. On trouve donc une correction pour la variance des OLS.
\end{itemize}
\end{block}
\end{frame}

\begin{frame}{Tests diagnostiques d'hétéroscédasticité}
\begin{block}{Estimateur de la variance suggéré}
\begin{align*}
\hat{V}(\hat{\beta}_{OLS})=T(X'X)^{-1} \hat{\Lambda} (X'X)^{-1}
\end{align*}
Où
\begin{align*}
\hat{\Lambda}=\frac{1}{T} \sum_{t=1}^T \hat{u}_t^2 (x_t'x_t)
\end{align*}
Où $x_t$ est la $t^{ième}$ ligne de la matrice X. Cet estimateur est difficile à visualiser. 
\begin{align*}
Var(\hat{\beta}_j)=\frac{\sum \hat{w}_{ji}^2\hat{u}_i^2}{(\sum \hat{w}_{ji}^2)^2}
\end{align*}
\end{block}
\end{frame}

\begin{frame}{Tests diagnostiques d'hétéroscédasticité}
\begin{block}{Estimateur de la variance suggéré}
\begin{itemize}
\item Les $\hat{u}_i$ sont les résidus de la régression originale et les $\hat{w}_j$ sont les résidus obtenus d'une régression auxiliaire du régresseur j sur tous les autres.
\item Le correcteur de White est donc une approximation de la variance qui pondère les observations des résidus en fonction des régresseurs dans le calcul de la variance. 
\item En général, si on a des indices sur la structure de la matrice $\Omega$, il est préférable de passer par l’estimateur FGLS si la qualité de l’information ajoutée est bonne. 
\item Le correcteur de White est très utile lorsqu’on n’a aucune idée de la forme fonctionnelle à donner à $\Omega$. 
\end{itemize}
\end{block}
\end{frame}
\end{document}
